\documentclass[14pt, titlepage, fleqn]{extarticle}
\usepackage[russian]{babel}
\usepackage{amsmath}
\usepackage{listings}
\usepackage{xcolor}

\title{Отчет к Лабораторной работе №1}
\author{Брызгалов Александр Витальевич\\Б8118-02.03.01сцт}
\date{3 Апреля 2020}

\begin{document}
	
	\maketitle
	
	\tableofcontents
	\addcontentsline{toc}{subsection}{Содержание}		
	
	\newpage
	\subsection*{Описать решение интеграла}
	\addcontentsline{toc}{subsection}{Описать решение интеграла}		

	1. Вычислить следующий интеграл с подробным описанием всех действий:
	$$ \int \frac{\ln x^2}{x^2} ~ dx$$
	
	\begin{equation*}
		\begin{aligned}
			\int \frac{\ln x^2}{x^2} ~ dx &= (*) \left \vert
				\begin{matrix}
					u = \ln x^2 & du = \frac{2}{x}~dx\\
					dv = \frac{1}{x^2} ~ dx  & v = -\frac{1}{x}
				\end{matrix}
			\right \vert (*) = -\frac{\ln x^2}{x} + \int \frac{dx}{x^2} = \\ 
			&= -\frac{\ln x^2 + 2}{x} + C
		\end{aligned}
	\end{equation*}

	\newpage
	\subsection*{Численное решение}

	2. Численно вычислить следующий интеграл с точностью $\epsilon = 10^{-3}$:

	$$\int\limits_1^\infty  \frac{{\rm e}^{-x}}{x} ~ dx$$

	\addcontentsline{toc}{subsection}{Численное решение}
	\subsection*{Метод левых прямоугольников}
	\begin{lstlisting}[language=C++, basicstyle=\footnotesize\ttfamily, frame = single]

double f(double x);
bool checkEnd(double F);

int main() {
	double answer = 0, 
	x = 1, 
	dx = 0.000001;
	bool counting = true;
	while(counting) {		
		double F = f(x);
		answer += F * dx;
		x += dx;
		counting = checkEnd(F);		
	}
	cout << answer;
}

double f(double x) { return exp(-x) / x; }

bool checkEnd(double F) {
	if (abs(F) < 0.0000000001) return false;
	else return true;
}
	
	\end{lstlisting}
		

	\newpage
	\subsection*{Метод правых прямоугольников}
	\begin{lstlisting}[language=C++, basicstyle=\footnotesize\ttfamily, frame = single]

double f(double x);
bool checkEnd(double F);

int main() {
	double answer = 0, 
	x = 1, 
	dx = 0.000001;
	bool counting = true;
	while(counting) {		
		double F = f(x + dx);
		answer += F * dx;
		x += dx;
		counting = checkEnd(F);		
	}
	cout << answer;
}

double f(double x) { return exp(-x) / x; }

bool checkEnd(double F) {
	if (abs(F) < 0.0000000001) return false;
	else return true;
}

	\end{lstlisting}

	\newpage
	\subsection*{Метод средних прямоугольников}
	\begin{lstlisting}[language=C++, basicstyle=\footnotesize\ttfamily, frame = single]

double f(double x);
bool checkEnd(double F);

int main() {
	double answer = 0, 
	x = 1, 
	dx = 0.000001;
	bool counting = true;
	while(counting) {		
		double F = f((x + dx + x) / 2);
		answer += F * dx;
		x += dx;
		counting = checkEnd(F);		
	}
	cout << answer;
}

double f(double x) { return exp(-x) / x; }

bool checkEnd(double F) {
	if (abs(F) < 0.0000000001) return false;
	else return true;
}

	\end{lstlisting}

	\newpage
	\subsection*{Метод трапеции}
	\begin{lstlisting}[language=C++, basicstyle=\footnotesize\ttfamily, frame = single]

double f(double x);
bool checkEnd(double F);

int main() {
	double answer = 0, 
	x = 1, 
	dx = 0.000001;
	bool counting = true;
	while(counting) {		
		double F = (f(x) + f(x + dx)) / 2;
		answer += F * dx;
		x += dx;
		counting = checkEnd(F);		
	}
	cout << answer;
}

double f(double x) { return exp(-x) / x; }

bool checkEnd(double F) {
	if (abs(F) < 0.0000000001) return false;
	else return true;
}

	\end{lstlisting}
\newpage
1. Метод левых прямоугольников:  0.219384, ~ $\delta$ = 0.119732\\

2. Метод правых прямоугольников: 0.219384, ~ $\delta$ = 0.119732\\

3. Метод средних прямоугольников: 0.219384, ~ $\delta$ 0.119732\\

4. Метод трапеции: 0.219384, ~ $\delta$ = 0.119732\\

5. Реальное решение: 0.219384

\newpage
	\subsection*{Тип дифференциальных уравнений и их общее решение}
	\addcontentsline{toc}{subsection}{Тип дифференциальных уравнений и их общее решение}

	$${\rm e}^{x} \cdot \sin^{3}{y} = y'\cdot \sin{x}$$ 
	Тип: уравнение с разделяющимися переменными.\\	
	Ответ: 
	\begin{equation}
		{\frac{\cos^3{y} - 3 \cos{y}}{3} = C - \frac{e^{-\log{e \cdot x}}(\log{e} \sin{x} + \cos{x})}{(\log{e})^2 + 1}}
	\end{equation}

	$$xy' \cdot \tg{y} = 2 - x^2 \cdot \ln(x^2 \cdot \cos{y})$$
	Тип: уравнение приводящееся к линейному.\\
	Ответ:
	\begin{equation}
		{y(x) = \pm \cos^{-1}{\left( \frac{e^{Ce^{x^2/2}}}{x^2} \right)}}
	\end{equation}

	$$y' = \frac{2x + 3y - 5}{5 - 3x - 2y}$$ 
	Тип: уравнение приводящееся к однородному.\\
	Ответ:
	\begin{equation}
		{y(x)= \frac{1}{2}(5 - 3x) \pm \frac{\sqrt{C + 4 (\frac {5x}{2} - \frac{x^2}{2}) + \frac{1}{2}(3x-5)^2}}{\sqrt{2}} + \frac{1}{2}(5 - 3x)}
	\end{equation}
	
	$$y' \cdot \tg{x} - y = x \cdot \ cos{x}$$ 
	Тип: линейное уравнение 1-го порядка.\\
	Ответ:
	\begin{equation}
		{y(x) = C \cdot \sin(x) - \frac{1}{2} \cdot x^2 \cdot \sin(x) - x\cos(x) + \sin(x) \cdot \ln(\sin(x))}
	\end{equation}

	\newpage
	\subsection*{Проверка решения задачи Коши}
	\addcontentsline{toc}{subsection}{Проверка решения задачи Коши}
		$$xy' \cdot (e^{y} - x), ~~~y(0) = \ln{2}; ~~~xy = e^y-2$$
		

\end{document}








